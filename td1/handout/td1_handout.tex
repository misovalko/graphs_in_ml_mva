\documentclass{article}
\usepackage{graphicx}
\usepackage{hyperref}
\usepackage{amsmath}
\usepackage[T1]{fontenc}
\usepackage{amsfonts}
\usepackage{enumitem}
\usepackage{natbib}
\usepackage{listings}
\setenumerate[1]{label=\thesection.\arabic*.}
\usepackage{datetime}

% Paragraph indent
\setlength\parindent{0pt}


%\usepackage{enumerate}
\newcommand{\wt}{\widetilde}
\newcommand{\wh}{\widehat}


\newcommand{\ba}{\bold{a}}
\newcommand{\bb}{\bold{b}}
\newcommand{\bc}{\bold{c}}
\newcommand{\bd}{\bold{d}}
\newcommand{\be}{\bold{e}}
\newcommand{\bof}{\bold{f}}
\newcommand{\bg}{\bold{g}}
\newcommand{\bh}{\bold{h}}
\newcommand{\bi}{\bold{i}}
\newcommand{\bj}{\bold{j}}
\newcommand{\bk}{\bold{k}}
\newcommand{\bl}{\bold{l}}
\newcommand{\bm}{\bold{m}}
\newcommand{\bn}{\bold{n}}
\newcommand{\bo}{\bold{o}}
\newcommand{\bp}{\bold{p}}
\newcommand{\bq}{\bold{q}}
\newcommand{\br}{\bold{r}}
\newcommand{\bs}{\bold{s}}
\newcommand{\bt}{\bold{t}}
\newcommand{\bu}{\bold{u}}
\newcommand{\bv}{\bold{v}}
\newcommand{\bw}{\bold{w}}
\newcommand{\bx}{\bold{x}}
\newcommand{\by}{\bold{y}}
\newcommand{\bz}{\bold{z}}
\newcommand{\bA}{\bold{A}}
\newcommand{\bB}{\bold{B}}
\newcommand{\bC}{\bold{C}}
\newcommand{\bD}{\bold{D}}
\newcommand{\oD}{\overline{D}}
\newcommand{\bE}{\bold{E}}
\newcommand{\bF}{\bold{F}}
\newcommand{\bG}{\bold{G}}
\newcommand{\bH}{\bold{H}}
\newcommand{\bI}{\bold{I}}
\newcommand{\bJ}{\bold{J}}
\newcommand{\bK}{\bold{K}}
\newcommand{\bL}{\bold{L}}
\newcommand{\bM}{\bold{M}}
\newcommand{\bN}{\bold{N}}
\newcommand{\bO}{\bold{O}}
\newcommand{\bP}{\bold{P}}
\newcommand{\bQ}{\bold{Q}}
\newcommand{\bR}{\bold{R}}
\newcommand{\bS}{\bold{S}}
\newcommand{\bT}{\bold{T}}
\newcommand{\bU}{\bold{U}}
\newcommand{\bV}{\bold{V}}
\newcommand{\bW}{\bold{W}}
\newcommand{\bX}{\bold{X}}
\newcommand{\oX}{\overline{X}}
\newcommand{\bY}{\bold{Y}}
\newcommand{\bZ}{\bold{Z}}
\newcommand{\calS}{\mathcal{S}}
%%%%%%%%%%%%%%%%%%%%%%%%%%%%
%%%%%%%%%%%%%%%%%%%%%%%%%%%%
%%%%%%%%%%%%%%%%%%%%%%%%%%%%
% definitions for hyperref %
%%%%%%%%%%%%%%%%%%%%%%%%%%%%
%fix counters
%\usepackage{aliascnt}

%%%%%%%%%%%%%%%%%%%%%%%%%%%%
% Optimization problems    %
%%%%%%%%%%%%%%%%%%%%%%%%%%%%
\newcommand{\probmtflprimal}{\mathcal{E}_{\mbox{\tiny MTFL}}}
\newcommand{\probmtfldual}{\mathcal{C}_{\mbox{\tiny MTFL}}}
\newcommand{\probmtfldualeps}{\mathcal{C}_{\mbox{$\varepsilon$-\tiny {MTFL}}}}
\newcommand{\mtflmodprimal}{\mathcal{E}}
\newcommand{\mtflmoddual}{\mathcal{C}}
\newcommand{\mtflmoddualeps}{{\mathcal{C}_{\vareps}}}
\newcommand{\mtflmoddualmerge}{{\mathcal{S}_{\vareps}}}
%%%%%%%%%%%%%%%%%%%%%%%%%%%%
% Algorithms stuff         %
%%%%%%%%%%%%%%%%%%%%%%%%%%%%
%\algnewcommand\algorithmicinput{\textbf{input:}}
%\algnewcommand\Input{\item[\algorithmicinput]}
%\algnewcommand\algorithmicoutput{\textbf{output:}}
%\algnewcommand\Output{\item[\algorithmicoutput]}
%\algdef{SE}[DOWHILE]{Do}{DoWhile}{\algorithmicdo}[1]{\algorithmicwhile\ #1}
%%%%%%%%%%%%%%%%%%%%%%%%%%%%
% Matrix operators         %
%%%%%%%%%%%%%%%%%%%%%%%%%%%%
\newcommand{\transp}{\mathsf{T}}
\DeclareMathOperator*{\argmin}{arg\,min}
\DeclareMathOperator*{\argmax}{arg\,max}
\DeclareMathOperator*{\trace}{trace}
\DeclareMathOperator*{\range}{Ran}
\DeclareMathOperator*{\kernel}{Ker}
\DeclareMathOperator*{\diag}{Diag}
\DeclareMathOperator*{\Vecm}{Vec}
\DeclareMathOperator*{\Rank}{Rank}
\DeclareMathOperator*{\Row}{Row}
\DeclareMathOperator*{\Col}{Col}
\DeclareMathOperator*{\Op}{\textbf{op}}
\DeclareMathOperator*{\nnz}{\textbf{nnz}}
%%%%%%%%%%%%%%%%%%%%%%%%%%%%
% Norms                    %
%%%%%%%%%%%%%%%%%%%%%%%%%%%%
\newcommand{\norm}[2]{\left\Vert #1 \right\Vert_{#2}}
\newcommand{\normsmall}[2]{\Vert #1 \Vert_{#2}}
%%%%%%%%%%%%%%%%%%%%%%%%%%%%
% Statistic operators      %
%%%%%%%%%%%%%%%%%%%%%%%%%%%%
\newcommand{\probability}{\mathbb{P}}
\newcommand{\probdist}{Pr}
\DeclareMathOperator*{\expectedvalue}{\mathbb{E}}
\newcommand{\expectedvalueover}[1]{\expectedvalue\limits_{#1}}
\newcommand{\gaussdistr}{\mathcal{N}}
\newcommand{\uniformdistr}{\mathcal{U}}
\newcommand{\indvec}{J}
%%%%%%%%%%%%%%%%%%%%%%%%%%%%
% Algebraic Sets           %
%%%%%%%%%%%%%%%%%%%%%%%%%%%%
\newcommand{\realnumbers}{\mathbb{R}}
\newcommand{\realnumberssdp}{\mathbb{R}_{+}}
\newcommand{\realnumbersdp}{\mathbb{R}_{++}}
\newcommand{\naturalnumbers}{\mathbb{N}}
\newcommand{\symmetricsdp}{\boldsymbol{S}^{d}_{+}}
\newcommand{\symmetricdp}{\boldsymbol{S}^{d}_{++}}
\newcommand{\symmetric}{\boldsymbol{S}^{d}}
\newcommand{\orthogonal}{\boldsymbol{O}^{d}}
%%%%%%%%%%%%%%%%%%%%%%%%%%%%
% MDP related notation     %
%%%%%%%%%%%%%%%%%%%%%%%%%%%%
\newcommand{\statespace}{\mathcal{X}}
\newcommand{\statespacedisc}{S}
\newcommand{\actionspace}{\mathcal{A}}
\newcommand{\transitionspace}{\mathcal{P}}
\newcommand{\transitionkernel}{P}
\newcommand{\rewardspace}{\mathcal{R}}
\newcommand{\rewardkernel}{R}
\newcommand{\epochset}{\mathcal{T}}
\newcommand{\trajspace}{\mathcal{H}}
\newcommand{\funcspace}{\mathcal{F}}
\newcommand{\dataset}{\mathcal{D}}
\newcommand{\Qfuncspace}{\mathcal{Q}}
\newcommand{\borelsetspace}{\mathcal{B}}
\newcommand{\kernelspace}{\mathcal{K}}
%\newcommand{\muset}{\boldsymbol\mu}
\newcommand{\muset}{\mu_{\pi}}
%\newcommand{\rhoset}{\boldsymbol\rho}
\newcommand{\rhoset}{\rho_{\pi}}
\newcommand{\rewardaverage}{R}
\newcommand{\rewardsample}{r}
\newcommand{\rewardsamplevec}{\br}
%\newcommand{\optbellop}{\mathcal{T}^{*}}
\newcommand{\optbellop}{\mathcal{T}}
\newcommand{\T}{\mathcal{T}}
\newcommand{\D}{\mathcal{P}}
\renewcommand{\Re}{\mathbb{R}}
%%%%%%%%%%%%%%%%%%%%%%%%%%%%
%%%%%%%%%%%%%%%%%%%%%%%%%%%%
%Short version of notation %
%%%%%%%%%%%%%%%%%%%%%%%%%%%%
\newcommand{\wrt}{w.r.t. }
%\newcommand{\defeq}{\stackrel{\mathclap{\normalfont\mbox{\tiny def}}}{=}}
\newcommand{\maxund}[1]{\max\limits_{#1}}
\newcommand{\minund}[1]{\min\limits_{#1}}
\newcommand{\vareps}{\varepsilon}
\newcommand{\bigotime}{\mathcal{O}}
%%%%%%%%%%%%%%%%%%%

%%%%%%%%%%%%%%%%%%%

%\newtheorem{theorem}{Theorem}
%\newtheorem{lemma}{Lemma}
%\newtheorem{corollary}{Corollary}
%\newtheorem{proposition}{Proposition}
%%\newtheorem{definition}{Definition}
%\newtheorem{assumption}{Assumption}
%%\newtheorem{remark}{Remark}

\newcommand{\gl}{{\sc GL}\xspace}
\newcommand{\mtfl}{{\sc MTFL}\xspace}
\newcommand{\fqi}{{\sc F$Q$I}\xspace}
\newcommand{\lassofqi}{{\sc LASSO--F$Q$I}\xspace}
\newcommand{\glfqi}{{\sc GL--F$Q$I}\xspace}
\newcommand{\flfqi}{{\sc FL--F$Q$I}\xspace}
\newcommand{\lasso}{{\sc LASSO}\xspace}



%\newaliascnt{lemma}{theorem}
%\newtheorem{lemma}[lemma]{Lemma}
%\aliascntresetthe{lemma}
%\providecommand*{\lemmaautorefname}{Lemma}
%\newaliascnt{corollary}{theorem}
%\newtheorem{corollary}[corollary]{Corollary}
%\aliascntresetthe{corollary}
%\providecommand*{\corollaryautorefname}{Corollary}
%\newaliascnt{proposition}{theorem}
%\newtheorem{proposition}[proposition]{Proposition}
%\aliascntresetthe{proposition}
%\providecommand*{\propositionautorefname}{Proposition}
%\newaliascnt{definition}{theorem}
%\newtheorem{definition}[definition]{Definition}

%\aliascntresetthe{definition}
%\providecommand*{\definitionautorefname}{Definition}
%\newaliascnt{assumption}{theorem}
%\newtheorem{assumption}[assumption]{Assumption}
%\aliascntresetthe{assumption}
%\providecommand*{\assumptionautorefname}{Assumption}


%\newenvironment{remark}[1][Remark]{\begin{trivlist}
%\item[\hskip \labelsep {\bfseries #1}]}{\end{trivlist}}



\setlength{\parskip}{\baselineskip}%

\usepackage{color}
\usepackage[defaultmono]{droidmono}

\definecolor{mygreen}{rgb}{0,0.6,0}
\definecolor{mygray}{rgb}{0.5,0.5,0.5}
\definecolor{mymauve}{rgb}{0.58,0,0.82}

\lstset{ %
backgroundcolor=\color{white},   % choose the background color; you must add \usepackage{color} or \usepackage{xcolor}
basicstyle=\fdmfamily\footnotesize,        % the size of the fonts that are used for the code
breakatwhitespace=false,         % sets if automatic breaks should only happen at whitespace
breaklines=true,                 % sets automatic line breaking
captionpos=b,                    % sets the caption-position to bottom
commentstyle=\color{mygreen},    % comment style
deletekeywords={...},            % if you want to delete keywords from the given language
escapeinside={\%*}{*)},          % if you want to add LaTeX within your code
extendedchars=true,              % lets you use non-ASCII characters; for 8-bits encodings only, does not work with UTF-8
frame=single,	                   % adds a frame around the code
keepspaces=true,                 % keeps spaces in text, useful for keeping indentation of code (possibly needs columns=flexible)
keywordstyle=\color{blue},       % keyword style
language=Octave,                 % the language of the code
otherkeywords={*,...},            % if you want to add more keywords to the set
numbers=left,                    % where to put the line-numbers; possible values are (none, left, right)
numbersep=5pt,                   % how far the line-numbers are from the code
numberstyle=\tiny\color{mygray}, % the style that is used for the line-numbers
rulecolor=\color{black},         % if not set, the frame-color may be changed on line-breaks within not-black text (e.g. comments (green here))
showspaces=false,                % show spaces everywhere adding particular underscores; it overrides 'showstringspaces'
showstringspaces=false,          % underline spaces within strings only
showtabs=false,                  % show tabs within strings adding particular underscores
stepnumber=2,                    % the step between two line-numbers. If it's 1, each line will be numbered
stringstyle=\color{mymauve},     % string literal style
tabsize=2,	                   % sets default tabsize to 2 spaces
title=\lstname                   % show the filename of files included with \lstinputlisting; also try caption instead of title
}



\begin{document}
\title{TP1: Spectral Clustering}
\author{pierre (dot) perrault (at) inria.fr \\ omar (dot) darwiche-domingues (at) inria.fr}
\date{October 15, 2019}
\maketitle

\begin{abstract}
    In this practical session we will cover fundamental graph building
    techniques, and apply them to the Spectral Clustering problem.
    The session will be evaluated on a short written report and a final
    image segmentation implementation. During
    the TD we will implement all the necessary tools to do the
    segmentation and answer the
    report questions.
    The report and the code are due in 2 weeks (deadline 23:59 29/11/2019).
    You will find instructions on how to submit the report on piazza,
    as well as the policies for scoring and late submissions.
    All the code related to the TD must be submitted,
    to provide background for the image segmentation code evaluation.

\end{abstract}

\newpage

%A small preface (especially for those that will not be present at the
%TD). Some of the experiments that will be presented during this TD makes use
%of randomly generated datasets. Because of this, there is always the
%possibility that a single run will be not representative of the
%usual outcome. Randomness in the data is common in Machine Learning,
%and managing this randomness is important. Proper experimental setting calls
%for repeated experiments and confidence intervals. In this case, it will be
%sufficient to repeat each experiment multiple times and visually see
%if there is large variations (some experiments are designed exactly to show
%this variations).


%Whenever the instructions require you to complete a file
%this means to open the corresponding
%file, and complete it according to the instructions all the following
%sections
%{\small
%\begin{lstlisting}
%
%%%%%%%%%%%%%%%%%%%%%%%%%%%%%%%%%%%%%%%%%%%%%%%%%%%%%%%%%%%%%%%%%%
%% use the build_similarity_graph function to build the graph W  %
%% W: (n x n) dimensional matrix representing                    %
%%    the adjacency matrix of the graph                          %
%%%%%%%%%%%%%%%%%%%%%%%%%%%%%%%%%%%%%%%%%%%%%%%%%%%%%%%%%%%%%%%%%%
%
%YOUR CODE HERE
%
%%%%%%%%%%%%%%%%%%%%%%%%%%%%%%%%%%%%%%%%%%%%%%%%%%%%%%%%%%%%%%%%%%
%%%%%%%%%%%%%%%%%%%%%%%%%%%%%%%%%%%%%%%%%%%%%%%%%%%%%%%%%%%%%%%%%%
%\end{lstlisting}
%}
%
%For each of these file the documentation at the top describes their parameters.


%If you use Matlab, before running any code, move to the root directory where
%\path{load_td.m} is located, and run it to properly set
%the required Matlab paths and import packages.

\newpage

\section{Graph construction}
The file \path{generate_data.py} contains functions that will generate artificial
data for the experiments, as described below. You can run the script to visualize the data.

\begin{enumerate}
    \item[] \textbf{$N$-Blob}: Sample random points in $\mathbb{R}^2$ according to $N$
        Gaussian distribution with mean
        $\mu_i = [\cos(2\pi i / N),\sin(2\pi i / N)]$ and variance $\mbox{Diag}(\sigma^2)$.
        As an example, 3-Blob is a dataset generated using this distribution
        and three clusters.
    \item[] \textbf{Two Moons}: Sample random points shaped as two intertwined moons.
    \item[] \textbf{Point and Circle}: Sample random points from a concentrated Gaussian
        point in the middle and a wide circle around it.
\end{enumerate}

%The first task of the session is to build and plot a grap representation of the
%data. In order to do this you will complete the \path{plot_similarity_graph.m} file
%with the necessary code.

A prerequisite to build a similarity graph is to define a similarity function
to score the distance between nodes in the graph.
For the rest of the session we will use an inverse exponential function
as the similarity measure, controlled by the euclidean distance.
$$ d(x_i, x_j) = \exp \left\{-\frac{\Vert x_i - x_j \Vert^2_2}{2\sigma^2}\right\}$$
The variance $\sigma^2$ of the Gaussian will control
the bandwidth of the similarity.

%You should now complete \path{build_similarity_graph.m}. In this file,
%you will write the code necessary to build an $\varepsilon$ graph and an
%(OR) $k$-nn graph. More details on these terms are contained in
%the source code of the file.

In the file \path{build_similarity_graph.py}, do the following: 
\begin{itemize}
	\item  Write the code to build an $\varepsilon$ graph and an
	(OR) $k$-nn graph. More details on these terms are contained in
	the source code of the file.
	\item Use the function \path{plot_similarity_graph} to visualize the graph for some generated data. The function \path{plot_graph_matrix}, in the file \path{utils.py} might also be useful.
	\item Complete the function \path{how_to_choose_epsilon}. You may use the function \path{min_span_tree} in \path{utils.py}.
\end{itemize}



Answer the following questions: 

\begin{enumerate}[resume]
	
    \item What is the purpose of the option parameter in \path{worst_case_blob} (in the file \path{generate_data.py})?
    \item  What happens when you change the generating parameter of
        \path{worst_case_blob} in \path{how_to_choose_espilon}
        and run the function? What if the parameter is very large?
    \item Using \path{plot_similarity_graph} and one of the datasets,
        compare $k$-nn to $\varepsilon$ graphs. When is it easier to build
        a connected graph using $k$-nn? When using $\varepsilon$ graphs?
\end{enumerate}


\section{Spectral Clustering}

In the file \path{spectral_clustering.py}, do the following: 

\begin{itemize}
	\item Complete the function \path{build_laplacian}.
	\item Complete the function \path{spectral_clustering}.
\end{itemize}



\begin{enumerate}
    \item Build a graph starting from the data generated in
        \path{two_blobs_clustering}, and remember to keep
        the graph connected.
        Motivate your choice on which eigenvectors to use
        and how you computed the clustering assignments from the eigenvectors.
        Now compute a similar clustering using the
        built-in $k$-means and compare the results.
    \item Build a graph starting from the data generated in
        \path{two_blobs_clustering}, but this time
        make it so that the two components are separate.
        How do you choose which eigenvectors to use in this case?
        Motivate your answer.
\end{enumerate}


So far we only considered clustering with $c=2$, and we did not establish
a systematic rule on how to select eigenvalues.  Do the following:

\begin{itemize}
	\item Complete the function \path{spectral_clustering_adaptive}, which is almost identical to \path{spectral_clustering}, except for the eigenvectors selection, which must use the function \path{choose_eig_function}.
\end{itemize}


\begin{enumerate}[resume]
    \item Look at the function \path{find_the_bend}.
        Generate a dataset with 4 blobs and
        $\sigma^2=0.03$. Build a graph out of it and plot
        the first 15 eigenvalues of the Laplacian.
        Complete the function \path{choose_eig_function}
        to automatically choose the number of eigenvectors to include.
        The decision rule must adapt to the actual eigenvalues of the
        problem.
    \item Now increase the variance of the Blobs to $\sigma^2=0.20$
        as you keep plotting
        the eigenvalues. Use the function \path{choose_eig_function}.
        Do you see any difference?
    \item When you built the cluster assignment, did you use thresholding,
        $k$-means or both? Do you have any opinion on when to use each?
    \item What is another use that looking at the distribution
        of the eigenvalues can have during clustering, beside
        choosing which eigenvectors to include?
\end{enumerate}

We will now consider more complex structures than Gaussian blobs. 

\begin{itemize}
	\item Complete the function \path{two_moons_clustering} 
	\item Complete the function \path{point_and_circle_clustering}.
\end{itemize}

\begin{enumerate}[resume]
    \item  Plot your results using spectral
        clustering and $k$-means in \path{two_moons_clustering} and compare the results.
        Do you notice any difference? Taking into consideration
        the graph structure, can you explain them?
    \item In the function \path{point_and_circle_clustering}, compare
        spectral clustering using the normal laplacian $L$ and the
        random-walk regularized Laplacian $L_{rw}$.
        Do you notice any difference? Taking into consideration
        the graph structure, can you explain them?
\end{enumerate}


How did you choose the parameters when building the graph?
We will compare clustering solutions while we change one of the parameters.
Evaluating clustering is not straightforward, since the final labeling
is arbitrary and not related to the original labels (in the rare cases
when talking about labels make sense). In particular, a labeling obtained
with a clustering algorithm might coincide with the true labels, but with
the label indices arbitrarily reshuffled (e.g. points in class $i$ labeled
as $j$ and points in class $j$ labeled as $i$).
To evaluate the assignment
we will use the Adjusted Rand Index score \cite{wagner2007comparing}.
The Rand Index takes as input two labelings of a dataset
and outputs a value between 0 (unrelated) and 1 (equal)
comparing how much these labeling overlap. The Adjusted RI
gives a better estimation for large number of clusters at the expense
of some (possibly unsatisfied) statistical assumption.
\begin{enumerate}[resume]
    \item Complete the function \path{parameter_sensitivity}, and
        generate a plot of the ARI index while varying one of the
        parameters in the graph construction $\varepsilon$ or $k$.
        Comment on the stability of spectral clustering.
    \item If we did not have access to \emph{true}\footnote{Definitions of
            true may vary.} labels how could we evaluate the clustering
        result (or what should we not use as evaluation)?
\end{enumerate}


\section{Image Segmentation}
Your final task, that you will complete after the class, is implementing
image segmentation using spectral clustering. As a simple example we will
segment images based on colors. In order to do this you must complete the file
\path{image_segmentation.py}. Here are some pointers (these are only suggestion)

\begin{itemize}
    \item The images provided are $50 \times 50$ pixels RGB images in the data folder. You
        need to specify them as input to the function \path{image_segmentation}.
        Inside the script, the images is loaded as a $50 \times 50 \times 3$ matrix,
        and converted into an $2500 \times 3$ matrix $X$
         where each pixel is a $\mathbb{R}^3$
        vector in RGB space.
    \item The image was chosen so that it could be easily segmented using
        colors. The easiest way is to build a graph based on color distance
        between individual pixels and cluster them.
\end{itemize}
The function \path{image_segmentation} will plot your segmentation. Because in clustering
the label are arbitrary the colors might be wrong. This does not matter as long
as the separation is correct.



Final questions:

\begin{enumerate}
    \item The first documentation is the code. If your code is well written
        and well commented, that is enough. If you think you need to better
        express some of the choices you made in the implementation,
        you can use this question in the report. Remember to include
        all the related code in the submission.
    \item A full graph built between the pixels of a $50 \times 50$ image
        corresponds to $50^2$ nodes. Solving the full eigenvalue problem
        in this case would scale in the order of $2^{34}$. Even on weak
        hardware (i.e. iPhone) this takes only seconds to minutes.
        Segmenting a Full HD picture of $1920 \times 1080$ would scale in the
        order of $2^{64}$ (about a month on a decent machine i.e. not an iPhone).
        Beyond that, the large picture would require
        to store in memory a graph over millions of nodes. A full graph
        on that scale requires about 1TB of memory. Can you think two simple
        techniques to reduce the computational and occupational cost of
        Spectral Clustering?
    \item Did you use \path{eig} or \path{eigs} to extract the final eigenvectors?
        Shortly, what is the difference between the two?
        How do they scale to large graphs (order of complexity)?
\end{enumerate}


\bibliographystyle{plain}
\bibliography{td}

\end{document}

